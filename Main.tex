\documentclass{article}
\usepackage[utf8]{inputenc}
\usepackage[italian] {babel}
\title{Scrittura, editing e divulgazione scientifica - Relazione}
\author{Gianantonio Volpe MLC/00644 }
\date{Maggio 2021}

\begin{document}

\maketitle
Nella seguente relazione tenterò di fare un sunto delle tematiche trattate e delle osservazioni estrapolate nel corso del ciclo di seminari
"Scrittura, editing e divulgazione scientifica", ospitati da Maydan sulla piattaforma Microsoft Teams, con un focus su quelle personalmente ritenute più interessanti. 
\\ Ho trovato estremamente stimolanti le possibilità fornite dall'utilizzo di un nuovo software di scrittura, quale LateX 
(con il quale ho peraltro deciso di compilare il testo in oggetto),
in quanto potenzialmente molto più funzionale alla formattazione di documenti professionali rispetto ai programmi normalmente impiegati per la redazione di articoli, 
saggi o documenti accademici: grazie alle sue capacità infatti risulta molto più agevole concentrarsi sul contenuto del documento che non sulla forma che gli si vuole
conferire, dal momento che sarà lo stesso LateX ad occuparsi dell'impaginazione e della sistematizzazione del testo; nonostante risulti inizialmente poco intuitivo, 
poiché necessita di input di comandi, con il semplice apprendimento di una serie di questi risulterà molto più comodo compilare un documento completo 
(con annessa bibliografia, per la quale esiste un apposito comando). Tornando invece alle questioni puramente teoriche affrontate nel corso dei diversi incontri,
la prima linea guida rilevante inerisce la scrittura di articoli scientifici, di modo che non vengano esclusi a priori: i prerequisiti includono una 
predisposizione alla lettura ed alla scrittura, per poi passare ad una serie di domande da porsi nel momento in cui si approccia l'atto concreto della redazione 
dell'articolo; tra queste vale la pena menzionare (e quindi chiedersi) se l'argomento che tratteremo non sia stato già preso in esame, da chi e come,
e se il contributo che vogliamo apportare differisce in qualche modo da quanto non sia già stato scritto. 
Fondamentale è l'impianto strutturale da dare al nostro paper: l'incipit deve consistere principalmente nella presentazione della rilevanza della
tematica e da una ricapitolazione succinta dello stato dell'arte, al fine di dimostrare conoscenza nell'ambito,
a cui può quindi far seguito lo sviluppo del contributo personale (meglio se introdotto da un breve riassunto delle tematiche trattate), 
aggiungendovi anche l'esemplificazione metodologica della raccolta dati; da non sottovalutare è anche l'apparato formale del testo,
ma è imprescindibile che il contenuto sia totalmente originale. Nella conclusione può risultare utile e sapiente fare un breve riassunto dell'articolo,
ma anche lasciare uno spiraglio per l'analisi dei problemi aperti e le futuribili ricerche. 
Ai fini della scrittura di articoli e documenti accademici è di strategica importanza anche un'oculata gestione della rassegna bibliografica:
nell'intraprendere il compito, bisogna innanzitutto identificare il tema, per poi raccogliere tutta la bibliografia possibile, 
organizzandola per tematiche e mettendo queste in relazione o in gruppi di contrasto; si può quindi proseguire con la descrizione del contenuto dei vari filoni 
identificati, mettendo particolare enfasi sui testi salienti, per finire con l'identificazione delle eventuali lacune (metodologiche o contenutistiche)
delle fonti esistenti. Per quanto concerne l'organizzazione tematica, la presentazione e la mappatura tematica, queste sono puramente arbitrarie, 
così come lo è la gerarchizzazione tematica; ciò che è rilevante, nel momento in cui si costruisce una bibliografia specifica, è partire dalle opere enciclopediche, 
per poi passare allo spoglio delle principali riviste del campo d'interesse. Nel momento in cui compiliamo la rassegna,
dobbiamo assicurarci che tutti i titoli citati figurino in essa e che il metodo di citazione sia coerente. Un'ultima valutazione da fare,
nel momento in cui determiniamo l'ampiezza della bibliografia, è quanto sia micro o macrografico il tema da affrontare per vagliare la possibilità di compilare una 
bibliografia sommaria o meno: se il tema è trasversale ed interdisciplinare, richiederà molte più fonti e richiamerà molta più attenzione da altri studiosi e vice 
versa. \\Altra tematica affrontata è quella delle recensioni, che si dividono in divulgative ed accademico-letterarie: le prime normalmente si distinguono per il 
titolo accattivante -fondamentale per l'adempimento del loro compito, ossia captare l'interesse del pubblico- e per essere fornite di una sinossi del testo; 
la recensione letteraria, d'altro canto, prevede un commento critico di uno studio, l'analisi delle sue debolezze e dei suoi punti di forza, 
oltre alla messa in relazione con lavori simili. Prima di iniziare il lavoro di redazione -critica- bisogna chiarire alcuni dei punti chiave, 
in maniera da facilitare l'organizzazione sistematica del lavoro, ad esempio: le idee chiave del tema di ricerca, la tesi principale e la struttura del lavoro,
lo sviluppo dell'argomento e le fonti consultate dall'autore, per finire con l'esplorazione delle conclusioni. Se invece ci stiamo occupando di una recensione di un 
testo narrativo, le specificità da ricercare nel momento in cui ci interpelliamo sono diverse. Se ci troviamo ad essere autori di una recensione, 
è imperativo il farlo in maniera quanto più oggettiva, meno umiliante e sarcastica possibile, senza cercare di stroncare né di incensare eccessivamente. \\ 
In conclusione, e per ricapitolare, mettendo in pratica e facendo interagire tutte queste indicazioni, dovrebbe essere possibile ricavare un testo strutturalmente 
ben formato, con un contenuto originale, con un'ottima formattazione e accademicamente rilevante, o quantomeno considerabile come tale. 
\end{document}
